\documentclass[]{article}
\usepackage{lmodern}
\usepackage{amssymb,amsmath}
\usepackage{ifxetex,ifluatex}
\usepackage{fixltx2e} % provides \textsubscript
\ifnum 0\ifxetex 1\fi\ifluatex 1\fi=0 % if pdftex
  \usepackage[T1]{fontenc}
  \usepackage[utf8]{inputenc}
\else % if luatex or xelatex
  \ifxetex
    \usepackage{mathspec}
	% 添加xeCJK宏包
    \usepackage{xeCJK}
    \setCJKmainfont[BoldFont=SimHei]{SimSun}
	%\setCJKmonofont{SimSun}% 设置缺省中文字体
    \setCJKsansfont{STFangsong}
    \setCJKmonofont{STXihei}
  \else
    \usepackage{fontspec}
  \fi
  % 注释掉此行,否则会冲突
  %\defaultfontfeatures{Ligatures=TeX,Scale=MatchLowercase}
\fi
% use upquote if available, for straight quotes in verbatim environments
\IfFileExists{upquote.sty}{\usepackage{upquote}}{}
% use microtype if available
\IfFileExists{microtype.sty}{%
\usepackage{microtype}
\UseMicrotypeSet[protrusion]{basicmath} % disable protrusion for tt fonts
}{}
\usepackage{hyperref}
\hypersetup{unicode=true,
            pdfborder={0 0 0},
            breaklinks=true}
\urlstyle{same}  % don't use monospace font for urls
\usepackage{color}
\usepackage{fancyvrb}
\newcommand{\VerbBar}{|}
\newcommand{\VERB}{\Verb[commandchars=\\\{\}]}
\DefineVerbatimEnvironment{Highlighting}{Verbatim}{commandchars=\\\{\}}
% Add ',fontsize=\small' for more characters per line
\newenvironment{Shaded}{}{}
\newcommand{\KeywordTok}[1]{\textcolor[rgb]{0.00,0.44,0.13}{\textbf{{#1}}}}
\newcommand{\DataTypeTok}[1]{\textcolor[rgb]{0.56,0.13,0.00}{{#1}}}
\newcommand{\DecValTok}[1]{\textcolor[rgb]{0.25,0.63,0.44}{{#1}}}
\newcommand{\BaseNTok}[1]{\textcolor[rgb]{0.25,0.63,0.44}{{#1}}}
\newcommand{\FloatTok}[1]{\textcolor[rgb]{0.25,0.63,0.44}{{#1}}}
\newcommand{\ConstantTok}[1]{\textcolor[rgb]{0.53,0.00,0.00}{{#1}}}
\newcommand{\CharTok}[1]{\textcolor[rgb]{0.25,0.44,0.63}{{#1}}}
\newcommand{\SpecialCharTok}[1]{\textcolor[rgb]{0.25,0.44,0.63}{{#1}}}
\newcommand{\StringTok}[1]{\textcolor[rgb]{0.25,0.44,0.63}{{#1}}}
\newcommand{\VerbatimStringTok}[1]{\textcolor[rgb]{0.25,0.44,0.63}{{#1}}}
\newcommand{\SpecialStringTok}[1]{\textcolor[rgb]{0.73,0.40,0.53}{{#1}}}
\newcommand{\ImportTok}[1]{{#1}}
\newcommand{\CommentTok}[1]{\textcolor[rgb]{0.38,0.63,0.69}{\textit{{#1}}}}
\newcommand{\DocumentationTok}[1]{\textcolor[rgb]{0.73,0.13,0.13}{\textit{{#1}}}}
\newcommand{\AnnotationTok}[1]{\textcolor[rgb]{0.38,0.63,0.69}{\textbf{\textit{{#1}}}}}
\newcommand{\CommentVarTok}[1]{\textcolor[rgb]{0.38,0.63,0.69}{\textbf{\textit{{#1}}}}}
\newcommand{\OtherTok}[1]{\textcolor[rgb]{0.00,0.44,0.13}{{#1}}}
\newcommand{\FunctionTok}[1]{\textcolor[rgb]{0.02,0.16,0.49}{{#1}}}
\newcommand{\VariableTok}[1]{\textcolor[rgb]{0.10,0.09,0.49}{{#1}}}
\newcommand{\ControlFlowTok}[1]{\textcolor[rgb]{0.00,0.44,0.13}{\textbf{{#1}}}}
\newcommand{\OperatorTok}[1]{\textcolor[rgb]{0.40,0.40,0.40}{{#1}}}
\newcommand{\BuiltInTok}[1]{{#1}}
\newcommand{\ExtensionTok}[1]{{#1}}
\newcommand{\PreprocessorTok}[1]{\textcolor[rgb]{0.74,0.48,0.00}{{#1}}}
\newcommand{\AttributeTok}[1]{\textcolor[rgb]{0.49,0.56,0.16}{{#1}}}
\newcommand{\RegionMarkerTok}[1]{{#1}}
\newcommand{\InformationTok}[1]{\textcolor[rgb]{0.38,0.63,0.69}{\textbf{\textit{{#1}}}}}
\newcommand{\WarningTok}[1]{\textcolor[rgb]{0.38,0.63,0.69}{\textbf{\textit{{#1}}}}}
\newcommand{\AlertTok}[1]{\textcolor[rgb]{1.00,0.00,0.00}{\textbf{{#1}}}}
\newcommand{\ErrorTok}[1]{\textcolor[rgb]{1.00,0.00,0.00}{\textbf{{#1}}}}
\newcommand{\NormalTok}[1]{{#1}}
\usepackage{longtable,booktabs}
\usepackage{graphicx,grffile}
\makeatletter
\def\maxwidth{\ifdim\Gin@nat@width>\linewidth\linewidth\else\Gin@nat@width\fi}
\def\maxheight{\ifdim\Gin@nat@height>\textheight\textheight\else\Gin@nat@height\fi}
\makeatother
% Scale images if necessary, so that they will not overflow the page
% margins by default, and it is still possible to overwrite the defaults
% using explicit options in \includegraphics[width, height, ...]{}
\setkeys{Gin}{width=\maxwidth,height=\maxheight,keepaspectratio}
\usepackage[normalem]{ulem}
% avoid problems with \sout in headers with hyperref:
\pdfstringdefDisableCommands{\renewcommand{\sout}{}}
\IfFileExists{parskip.sty}{%
\usepackage{parskip}
}{% else
\setlength{\parindent}{0pt}
\setlength{\parskip}{6pt plus 2pt minus 1pt}
}
\setlength{\emergencystretch}{3em}  % prevent overfull lines
\providecommand{\tightlist}{%
  \setlength{\itemsep}{0pt}\setlength{\parskip}{0pt}}
\setcounter{secnumdepth}{0}
% Redefines (sub)paragraphs to behave more like sections
\ifx\paragraph\undefined\else
\let\oldparagraph\paragraph
\renewcommand{\paragraph}[1]{\oldparagraph{#1}\mbox{}}
\fi
\ifx\subparagraph\undefined\else
\let\oldsubparagraph\subparagraph
\renewcommand{\subparagraph}[1]{\oldsubparagraph{#1}\mbox{}}
\fi

\date{}

\begin{document}

\section{h1 Heading 8-)}\label{h1-heading-8-}

\subsection{h2 Heading}\label{h2-heading}

\subsubsection{h3 Heading}\label{h3-heading}

\paragraph{h4 Heading}\label{h4-heading}

\subparagraph{h5 Heading}\label{h5-heading}

h6 Heading

\subsection{MathJax}\label{mathjax}

\(\sum_{i=0}^n i^2 = \frac{(n^2+n)(2n+1)}{6}\)
\[\sum_{i=0}^n i^2 = \frac{(n^2+n)(2n+1)}{6}\]

\subsection{Large paragraphs}\label{large-paragraphs}

Lorem ipsum dolor sit amet, consectetuer adipiscing elit. Pellentesque
pretium lectus id turpis. Cum sociis natoque penatibus et magnis dis
parturient montes, nascetur ridiculus mus. Fusce dui leo, imperdiet in,
aliquam sit amet, feugiat eu, orci. Aliquam erat volutpat. In rutrum.
Etiam dictum tincidunt diam. Integer pellentesque quam vel velit. Ut
enim ad minima veniam, quis nostrum exercitationem ullam corporis
suscipit laboriosam, nisi ut aliquid ex ea commodi consequatur? Etiam
bibendum elit eget erat. In enim a arcu imperdiet malesuada. Integer
lacinia. Duis condimentum augue id magna semper rutrum. In laoreet,
magna id viverra tincidunt, sem odio bibendum justo, vel imperdiet
sapien wisi sed libero.

Class aptent taciti sociosqu ad litora torquent per conubia nostra, per
inceptos hymenaeos. Fusce dui leo, imperdiet in, aliquam sit amet,
feugiat eu, orci. Mauris suscipit, ligula sit amet pharetra semper, nibh
ante cursus purus, vel sagittis velit mauris vel metus. Duis ante orci,
molestie vitae vehicula venenatis, tincidunt ac pede. Maecenas
sollicitudin. Nullam faucibus mi quis velit. Integer vulputate sem a
nibh rutrum consequat. Proin mattis lacinia justo. Suspendisse nisl.
Aenean placerat. Integer malesuada. Donec quis nibh at felis congue
commodo. Duis condimentum augue id magna semper rutrum. Duis bibendum,
lectus ut viverra rhoncus, dolor nunc faucibus libero, eget facilisis
enim ipsum id lacus.

\subsection{Horizontal Rules}\label{horizontal-rules}

\begin{center}\rule{0.5\linewidth}{\linethickness}\end{center}

\begin{center}\rule{0.5\linewidth}{\linethickness}\end{center}

\begin{center}\rule{0.5\linewidth}{\linethickness}\end{center}

\subsection{Typographic replacements}\label{typographic-replacements}

Enable typographer option to see result.

\begin{enumerate}
\def\labelenumi{(\alph{enumi})}
\setcounter{enumi}{2}
\item
  \begin{enumerate}
  \def\labelenumii{(\Alph{enumii})}
  \setcounter{enumii}{2}
  \item
    \begin{enumerate}
    \def\labelenumiii{(\alph{enumiii})}
    \setcounter{enumiii}{17}
    \item
      \begin{enumerate}
      \def\labelenumiv{(\Alph{enumiv})}
      \setcounter{enumiv}{17}
      \tightlist
      \item
        (tm) (TM) (p) (P) +-
      \end{enumerate}
    \end{enumerate}
  \end{enumerate}
\end{enumerate}

test.. test\ldots{} test\ldots{}.. test?\ldots{}.. test!\ldots{}.

!!!!!! ???? ,, -- ---

``Smartypants, double quotes'' and `single quotes'

\subsection{Emphasis}\label{emphasis}

\textbf{This is bold text}

\textbf{This is bold text}

\emph{This is italic text}

\emph{This is italic text}

\sout{Strikethrough}

\subsection{Blockquotes}\label{blockquotes}

\begin{quote}
Blockquotes can also be nested\ldots{} \textgreater{} \ldots{}by using
additional greater-than signs right next to each other\ldots{}
\textgreater{} \textgreater{} \ldots{}or with spaces between arrows.
\end{quote}

\subsection{Lists}\label{lists}

Unordered

\begin{itemize}
\tightlist
\item
  Create a list by starting a line with \texttt{+}, \texttt{-}, or
  \texttt{*}
\item
  Sub-lists are made by indenting 2 spaces:
\item
  Marker character change forces new list start:

  \begin{itemize}
  \tightlist
  \item
    Ac tristique libero volutpat at
  \item
    Facilisis in pretium nisl aliquet
  \item
    Nulla volutpat aliquam velit
  \end{itemize}
\item
  Very easy!
\end{itemize}

Ordered

\begin{enumerate}
\def\labelenumi{\arabic{enumi}.}
\item
  Lorem ipsum dolor sit amet
\item
  Consectetur adipiscing elit
\item
  Integer molestie lorem at massa
\item
  You can use sequential numbers\ldots{}
\item
  \ldots{}or keep all the numbers as \texttt{1.}
\end{enumerate}

Start numbering with offset:

\begin{enumerate}
\def\labelenumi{\arabic{enumi}.}
\setcounter{enumi}{56}
\tightlist
\item
  foo
\item
  bar
\end{enumerate}

\subsection{Code}\label{code}

Inline \texttt{code}

Indented code

\begin{verbatim}
// Some comments
line 1 of code
line 2 of code
line 3 of code
\end{verbatim}

Block code ``fences''

\begin{verbatim}
Sample text here...
\end{verbatim}

Syntax highlighting

\begin{Shaded}
\begin{Highlighting}[]
\KeywordTok{var} \NormalTok{foo }\OperatorTok{=} \KeywordTok{function} \NormalTok{(bar) }\OperatorTok{\{}
  \ControlFlowTok{return} \NormalTok{bar}\OperatorTok{++;}
\OperatorTok{\};}

\VariableTok{console}\NormalTok{.}\AttributeTok{log}\NormalTok{(}\AttributeTok{foo}\NormalTok{(}\DecValTok{5}\NormalTok{))}\OperatorTok{;}
\end{Highlighting}
\end{Shaded}

\subsection{Tables}\label{tables}

\begin{longtable}[]{@{}ll@{}}
\toprule
\begin{minipage}[b]{0.09\columnwidth}\raggedright\strut
Option\strut
\end{minipage} & \begin{minipage}[b]{0.16\columnwidth}\raggedright\strut
Description\strut
\end{minipage}\tabularnewline
\midrule
\endhead
\begin{minipage}[t]{0.09\columnwidth}\raggedright\strut
data\strut
\end{minipage} & \begin{minipage}[t]{0.16\columnwidth}\raggedright\strut
path to data files to supply the data that will be passed into
templates.\strut
\end{minipage}\tabularnewline
\begin{minipage}[t]{0.09\columnwidth}\raggedright\strut
engine\strut
\end{minipage} & \begin{minipage}[t]{0.16\columnwidth}\raggedright\strut
engine to be used for processing templates. Handlebars is the
default.\strut
\end{minipage}\tabularnewline
\begin{minipage}[t]{0.09\columnwidth}\raggedright\strut
ext\strut
\end{minipage} & \begin{minipage}[t]{0.16\columnwidth}\raggedright\strut
extension to be used for dest files.\strut
\end{minipage}\tabularnewline
\bottomrule
\end{longtable}

Right aligned columns

\begin{longtable}[]{@{}rr@{}}
\toprule
\begin{minipage}[b]{0.10\columnwidth}\raggedleft\strut
Option\strut
\end{minipage} & \begin{minipage}[b]{0.17\columnwidth}\raggedleft\strut
Description\strut
\end{minipage}\tabularnewline
\midrule
\endhead
\begin{minipage}[t]{0.10\columnwidth}\raggedleft\strut
data\strut
\end{minipage} & \begin{minipage}[t]{0.17\columnwidth}\raggedleft\strut
path to data files to supply the data that will be passed into
templates.\strut
\end{minipage}\tabularnewline
\begin{minipage}[t]{0.10\columnwidth}\raggedleft\strut
engine\strut
\end{minipage} & \begin{minipage}[t]{0.17\columnwidth}\raggedleft\strut
engine to be used for processing templates. Handlebars is the
default.\strut
\end{minipage}\tabularnewline
\begin{minipage}[t]{0.10\columnwidth}\raggedleft\strut
ext\strut
\end{minipage} & \begin{minipage}[t]{0.17\columnwidth}\raggedleft\strut
extension to be used for dest files.\strut
\end{minipage}\tabularnewline
\bottomrule
\end{longtable}

Multiple options

\begin{longtable}[]{@{}lcr@{}}
\toprule
Tables & Are & Cool\tabularnewline
\midrule
\endhead
col 1 is & left-aligned & \$1600\tabularnewline
col 2 is & centered & \$12\tabularnewline
col 3 is & right-aligned & \$1\tabularnewline
\bottomrule
\end{longtable}

\subsection{Links}\label{links}

\href{http://dev.nodeca.com}{link text}

\href{http://nodeca.github.io/pica/demo/}{link with title}

Autoconverted link https://github.com/nodeca/pica (enable linkify to
see)

\subsection{Images}\label{images}

\includegraphics{https://octodex.github.com/images/minion.png}
\includegraphics{https://octodex.github.com/images/stormtroopocat.jpg}

Like links, Images also have a footnote style syntax

\begin{figure}[htbp]
\centering
\includegraphics{https://octodex.github.com/images/dojocat.jpg}
\caption{Alt text}
\end{figure}

With a reference later in the document defining the URL location:

\subsection{Plugins}\label{plugins}

The killer feature of \texttt{markdown-it} is very effective support of
\href{https://www.npmjs.org/browse/keyword/markdown-it-plugin}{syntax
plugins}.

\subsubsection{\texorpdfstring{\href{https://github.com/markdown-it/markdown-it-emoji}{Emojies}}{Emojies}}\label{emojies}

\begin{quote}
Classic markup: :wink: :crush: :cry: :tear: :laughing: :yum:

Shortcuts (emoticons): :-) :-( 8-) ;)
\end{quote}

see
\href{https://github.com/markdown-it/markdown-it-emoji\#change-output}{how
to change output} with twemoji.

\subsubsection{\texorpdfstring{\href{https://github.com/markdown-it/markdown-it-sub}{Subscript}
/
\href{https://github.com/markdown-it/markdown-it-sup}{Superscript}}{Subscript / Superscript}}\label{subscript-superscript}

\begin{itemize}
\tightlist
\item
  19\textsuperscript{th}
\item
  H\textsubscript{2}O
\end{itemize}

\subsubsection{\texorpdfstring{\href{https://github.com/markdown-it/markdown-it-ins}{\textless{}ins\textgreater{}}}{\textless{}ins\textgreater{}}}\label{ins}

++Inserted text++

\subsubsection{\texorpdfstring{\href{https://github.com/markdown-it/markdown-it-mark}{\textless{}mark\textgreater{}}}{\textless{}mark\textgreater{}}}\label{mark}

==Marked text==

\subsubsection{\texorpdfstring{\href{https://github.com/markdown-it/markdown-it-footnote}{Footnotes}}{Footnotes}}\label{footnotes}

Footnote 1 link\footnote{Footnote \textbf{can have markup}

  and multiple paragraphs.}.

Footnote 2 link\footnote{Footnote text.}.

Inline footnote\footnote{Text of inline footnote} definition.

Duplicated footnote reference\footnote{Footnote text.}.

\subsubsection{\texorpdfstring{\href{https://github.com/markdown-it/markdown-it-deflist}{Definition
lists}}{Definition lists}}\label{definition-lists}

\begin{description}
\item[Term 1]
Definition 1 with lazy continuation.
\item[Term 2 with \emph{inline markup}]
Definition 2

\begin{verbatim}
{ some code, part of Definition 2 }
\end{verbatim}

Third paragraph of definition 2.
\end{description}

\emph{Compact style:}

\begin{description}
\tightlist
\item[Term 1]
Definition 1
\item[Term 2]
Definition 2a

Definition 2b
\end{description}

\subsubsection{\texorpdfstring{\href{https://github.com/markdown-it/markdown-it-abbr}{Abbreviations}}{Abbreviations}}\label{abbreviations}

This is HTML abbreviation example.

It converts ``HTML'', but keep intact partial entries like
``xxxHTMLyyy'' and so on.

*{[}HTML{]}: Hyper Text Markup Language

\subsubsection{\texorpdfstring{\href{https://github.com/markdown-it/markdown-it-container}{Custom
containers}}{Custom containers}}\label{custom-containers}

::: warning \emph{here be dragons} :::

\end{document}
